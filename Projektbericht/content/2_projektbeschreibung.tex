\chapter{Projektbeschreibung}\label{ch:projektbeschreibung}
In den folgenden Abschnitten werden hauptsächlich, die anfänglichen Ideen und die daraus schließenden Projektziele und die dafür verwendete Vorgehensweise beschrieben.

\section{Ideenfindung}\label{ideenfindung}
Wie jedes Projekt beginnt auch dieses mit der Ideenfindung. Dabei war es unser Ziel, zu einem Ergebnis zu kommen, mit dem alle Teammitglieder zufrieden sind. 

Um eine Auswahl der Ideen aller Mitglieder zu erhalten, wurde die Brainstorming-Methode genutzt. Der Sinn und Zweck eines Brainstormings ist es natürlich so viele Ideen wie möglich zu generieren. Die unterschiedlichen Meinungen und Erfahrungen aller Mitglieder helfen dabei eine große Variation an Vorschlägen zu erhalten. Der Vorteil dabei liegt darin, dass alle möglichen Ideen eingebracht werden können, ohne sich dabei Gedanken machen zu müssen, ob man dabei kritisiert wird. 

Zu Beginn des Brainstormings, haben sich alle Mitglieder 15 Minuten Zeit genommen, um ihre Ideen niederzuschreiben. Da nicht jeder Vorschlag umsetzbar ist, musste nach einer kurzen Vorstellung der Ideen aussortiert werden. Dabei wurden wichtige Aspekte wie die Zustimmung der Teammitglieder und Umfang des Vorschlags beachtet. Das Team einigte sich dabei auf folgende Interessengebiete: Microcontroller, Augmented Reality und Appentwicklung.

Nach weiterem Abwegen der Ideen kamen wir zu dem Entschluss, die Appentwicklung mit Augmented Reality zu kombinieren. Dabei stand von Anfang an fest, dass es sich bei dem Appentwicklungsbereich um eine Lernapp in dem Fachbereich Geografie handeln soll. Um es interessanter und spannender für den Nutzer zu gestalten, soll der Augmented Reality Bereich in Form eines Spiels eingebunden werden. 

Damit stand eine grobe Aufbau für das Projekt \textquote{TraWo} , was für Travel The World steht, fest.

\section{Projektziel}
\label{projektziel}
Nachdem in Abschnitt \ref{ideenfindung} ein grober Aufbau der Anwendung beschrieben wurde, ist es nun an der Reihe die Projektziele festzulegen. Es war uns wichtig, die Anforderungen von Projektbeginn an zu bestimmen, um so eine klares Ziel anstreben zu können.

Die Anwendung ist hauptsächlich für Grundschüler gedacht und soll durch ein ansprechendes Design und einen interessanten Lernstil als Motivation für die Kinder dienen.

Wie bereits im vorherigen Abschnitt erwähnt soll TraWo in einen App- und Augmented Reality Bereich aufgeteilt werden. Der App Bereich soll zur Wiedergabe der Informationen dienen. Durch Erlernen dieser Inhalte soll der Nutzer dazu in der Lage sein den Spielteil der Anwendung zu absolvieren. Der Spiel Bereich besteht zum Teil aus dem Augmented Reality Teil und bietet die Möglichkeit, Marker mit einem Handy oder Tablet zu scannen. Die Marker können von den Nutzern ausgedruckt und auf einer Landkarte oder einen Globus angebracht werden. Auf diesen sollen unterschiedliche 3D Objekte angezeigt werden. Durch diese soll es möglich sein an einem Quiz über das jeweilige Land teilzunehmen.

Ein weiteres und wichtiges Ziel ist es die Kinder beim Nutzen der Anwendung motiviert zu halten. Dabei hilft bereits die Gelegenheit 3D Objekte auf einen Globus projizieren zu können, um zu dem Test zu gelangen. Diese Methode ist für die meisten Kindern noch unbekannt und bietet etwas Aufregendes und Spannendes. Eine weitere Alternative die Kinder zum Nutzen der App zu animieren ist ein Levelsystem. Dabei sollen alle Kontinent bis auf eins noch nicht spielbar sein. Die Freischaltung eines weiteren Kontinents ist erst nach erfolgreichem Abschluss des vorherigen möglich. Ein Kontinent gilt erst dann als abgeschlossen wenn alle enthaltenen Test bestanden wurden. Die dadurch entstehenden Erfolgsgefühle treiben den Nutzer zum Fortfahren an.

Um den Anforderungen eines IT-Projekts gerecht zu werden, soll die Entwicklung der App generisch gestaltet werden. Durch das Ändern der Inhalte soll es möglich sein das Produkt auch auf andere Anwendungsgebiete übertragen zu können.

\section{Vorgehensweise}
Bereits vor dem eigentlichen Projektstart hatten wir einen klaren Blick auf das fertige Produkt.
Mit der Wahl des Vorgehensmodells suchten wir also nur noch die schnellste und einfachste Art und Weise, diese Produktvision umzusetzen. 
So kamen agile Modelle wie beispielsweise \textquote{Scrum} gar nicht erst in Frage, da sie sich darauf spezialisieren immer wieder Platz für neue Anforderungen zu schaffen. 
Diese standen jedoch von Anfang an fest.
Somit war ein geradliniger und sequentieller Ablauf von Vorteil, welcher beim klassischen \textquote{Wasserfallmodell} gegeben ist. 

Hierbei wird das Projekt von Anfang bis Ende durchstrukturiert und ein genauer Ablaufplan befolgt.
Dieser entsteht vor dem Beginn des Projekts und umfasst mehrere Phasen. 
Unsere Aufteilung bestand aus fünf Phasen der Entwicklung und ist in Form eines Meilensteinplans (Abschnitt \ref{Meilensteinplan}) niedergeschrieben. 

Die erste Phase galt der Anforderungsanalyse. 
Wie bereits erwähnt hatten wir zu Anfang eine genaue Vorstellung der Anwendung und diese musste nun in eine anschauliche Form gebracht werden. 
Weiter ging es mit der Evaluation der Technologien und Methoden, beispielsweise die Datenhaltung innerhalb der App oder die Wahl der AR-Schnittstelle.
Der dritte Meilenstein umfasste das Aufsetzen der Entwicklungsumgebung und Versionsverwaltung, sowie die Aufteilung der Zuständigkeiten.
Im nächsten Schritt wurde das Frontend der Anwendung in Form von GUI-Mockups entworfen und das Backend mithilfe von Klassendiagrammen strukturiert.
In der letzten Phase widmeten wir uns schließlich der Implementierung.
