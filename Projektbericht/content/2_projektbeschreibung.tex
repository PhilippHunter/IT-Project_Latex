\chapter{Projektbeschreibung}\label{ch:projektbeschreibung}
In den folgenden Abschnitten werden hauptsächlich, die anfänglichen Ideen und die daraus schließenden Projektziele und die dafür verwendete Vorgehensweise beschrieben.

\section{Ideenfindung}
\section{Projektziel}\label{projektziel}
\section{Vorgehensweise}
Bereits vor dem eigentlichen Projektstart hatten wir einen klaren Blick auf das fertige Produkt.
Mit der Wahl des Vorgehensmodells suchten wir also nur noch die schnellste und einfachste Art und Weise, diese Produktvision umzusetzen. 
So kamen agile Modelle wie beispielsweise \textquote{Scrum} gar nicht erst in Frage, da sie sich darauf spezialisieren immer wieder Platz für neue Anforderungen zu schaffen. 
Diese standen jedoch von Anfang an fest.
Somit war ein geradliniger und sequentieller Ablauf von Vorteil, welcher beim klassischen \textquote{Wasserfallmodell} gegeben ist. 

Hierbei wird das Projekt von Anfang bis Ende durchstrukturiert und ein genauer Ablaufplan befolgt.
Dieser entsteht vor dem Beginn des Projekts und umfasst mehrere Phasen. 
Unsere Aufteilung bestand aus fünf Phasen der Entwicklung und ist in Form eines Meilensteinplans (Abschnitt \ref{Meilensteinplan}) niedergeschrieben. 

Die erste Phase galt der Anforderungsanalyse. 
Wie bereits erwähnt hatten wir zu Anfang eine genaue Vorstellung der Anwendung und diese musste nun in eine anschauliche Form gebracht werden. 
Weiter ging es mit der Evaluation der Technologien und Methoden, beispielsweise die Datenhaltung innerhalb der App oder die Wahl der AR-Schnittstelle.
Der dritte Meilenstein umfasste das Aufsetzen der Entwicklungsumgebung und Versionsverwaltung, sowie die Aufteilung der Zuständigkeiten.
Im nächsten Schritt wurde das Frontend der Anwendung in Form von GUI-Mockups entworfen und das Backend mithilfe von Klassendiagrammen strukturiert.
In der letzten Phase widmeten wir uns schließlich der Implementierung.
