\chapter{Projektbeschreibung}\label{ch:projektbeschreibung}
In den folgenden Abschnitten werden hauptsächlich die anfänglichen Ideen, die daraus schließenden Projektziele und die dafür verwendete Vorgehensweise beschrieben.

\section{Ideenfindung}\label{ideenfindung}
Gemäß eines gewohnten Projektablaufs, beginnt auch dieses mit der Ideenfindung. Unser Ziel war es, auf ein Ergebnis zu kommen, mit dem alle Teammitglieder zufrieden sind.

Um eine Auswahl der Ideen aller Mitglieder zu erhalten, wurde die Brainstorming-Methode genutzt. Der Sinn und Zweck eines Brainstormings ist es so viele Ideen wie möglich zu generieren. Die unterschiedlichen Meinungen und Erfahrungen aller Mitglieder helfen dabei eine große Variation an Vorschlägen zu erhalten. Der Vorteil dabei liegt darin, dass alle möglichen Ideen eingebracht werden können, ohne sich dabei Gedanken machen zu müssen kritisiert zu werden.

Zu Beginn des Brainstormings, haben sich alle Mitglieder 15 Minuten Zeit genommen, um ihre Ideen niederzuschreiben. Da nicht jeder Vorschlag umsetzbar ist, musste nach einer kurzen Vorstellung der Ideen aussortiert werden. Dabei wurden wichtige Aspekte, wie die Zustimmung der Teammitglieder und der Umfang des Vorschlags beachtet. Das Team einigte sich dabei auf folgende Interessengebiete: Microcontroller, Augmented Reality und Appentwicklung.

Nach weiterem Abwägen der Ideen kamen wir zu dem Entschluss, die zwei Themenblöcke Appentwicklung und Augmented Reality zu kombinieren. Dabei stand von Anfang an der Anwendungsfall einer Lernapp fest, bei deren Fachbereich sich auf die Geografie geeinigt werden konnte. Um es interessanter und spannender für den Nutzer zu gestalten, soll der Augmented Reality Bereich in Form eines Spiels eingebunden werden. 

Damit stand ein grobe Aufbau für das Projekt \textquote{TraWo} , was für \textquote{Travel The World} steht, fest.

\section{Projektziel}\label{projektziel}
Nachdem in Abschnitt \ref{ideenfindung} die Idee hinter der Anwendung beschrieben wurde, ist es nun an der Reihe die Projektziele festzulegen. Es war uns wichtig, die Anforderungen von Projektbeginn an zu bestimmen, um so ein klares Ziel anstreben zu können.

Die Anwendung ist zwar für mehrere Zielgruppen gedacht, richtet sich jedoch aber hauptsächlich an Grundschüler und soll durch ein ansprechendes Design und einen interessanten Lernstil als Motivation für die Kinder dienen.

Für uns stand von Anfang an fest, dass es sich bei TraWo um eine Android Anwendung handeln wird. Wie bereits im vorherigen Abschnitt erwähnt, soll diese in einen Informations- und einen Spielbereich aufgeteilt werden. Der Informationsteil soll zur Wiedergabe der Länderinformationen dienen. Durch Erlernen dieser Inhalte soll der Nutzer dazu in der Lage sein, den Spielteil der Anwendung zu absolvieren. Der Spielbereich besteht zum Teil aus dem Augmented Reality Teil und bietet die Möglichkeit, Marker mit einem Handy oder Tablet zu scannen. Dadurch können auf dem Bildschirm des Geräts 3D Objekte angezeigt werden, durch die es möglich sein soll ein länderspezifisches Quiz zu starten. Diese Marker können vom Nutzer ausgedruckt und auf einer Landkarte oder einem Globus angebracht werden.

Ein weiteres Ziel ist es, die Kinder beim Nutzen der Anwendung motiviert zu halten. Dafür soll die Gelegenheit, 3D Objekte auf einem Globus projizieren zu können sorgen. Diese Methode ist für die meisten Kinder noch unbekannt und bietet etwas Aufregendes und Spannendes. Eine weitere Alternative die Kinder zum Nutzen der App zu animieren ist ein Levelsystem. Dabei soll zu Beginn an nur der Heimatkontinent spielbar sein. Die Freischaltung eines weiteren Kontinents ist erst nach erfolgreichem Abschluss des vorherigen möglich. Ein Kontinent gilt erst dann als abgeschlossen, wenn alle enthaltenen Test bestanden wurden. Die dadurch entstehenden Erfolgsgefühle treiben den Nutzer zum Weiterspielen an.

Um den Anforderungen eines IT-Projekts gerecht zu werden, soll die Entwicklung der App generisch gestaltet werden. Durch das Ändern der Inhalte soll es möglich sein das Produkt auch auf andere Anwendungsgebiete übertragen zu können.

\section{Vorgehensweise}
Wie Abschnitt \ref{ideenfindung} und \ref{projektziel} zeigen, hatten wir bereits vor dem eigentlichen Projektstart einen klaren Blick auf das fertige Produkt.
Mit der Wahl des Vorgehensmodells suchten wir also nur noch die schnellste und einfachste Art und Weise, diese Produktvision umzusetzen. 
Agile Modelle wie beispielsweise \textquote{Scrum} kamen aus mehreren Gründen nicht in Frage. 
Zum einen gibt es bei Scrum eine feste Rollenverteilung. 
Es sieht vor, das Entwicklerteam mit einem Scrum-Master zu unterstützen, welcher die regelmäßigen Meetings koordiniert. 
Die Rolle des Product Owners stellt den Kunden dar, der sich um die Anforderungsanalyse und das Planen der Entwicklungsiterationen kümmert. 
Er sorgt dafür, dass dem Team immer ausreichend viele Aufgaben für die aktuelle Iteration zur Verfügung stehen. 
Eine sinnvolle Besetzung dieser beiden Rollen konnten im Rahmen des IT-Projekts nicht durchgeführt werden, was gegen die Entwicklung nach einem agilen Vorgehensmodell sprach.
Zum anderen erfolgt der Ablauf bei Scrum in einzelnen Iterationen, in welchen immer wieder Platz für neue Anforderungen geschaffen wird. 
Unsere Anforderungen standen von Beginn an fest und somit entstand diesbezüglich kein Bedarf für eine agile Herangehensweise.

Der geradlinige und sequentielle Ablauf, der beim klassischen \textquote{Wasserfallmodell} gegeben ist, passte mehr zu unserem gesuchten Vorgehensmodell.
Hierbei wird das Projekt von Anfang bis Ende durchstrukturiert und ein genauer Ablaufplan befolgt.
Dieser entsteht vor dem Beginn des Projekts und umfasst mehrere Phasen. 
Unsere Aufteilung bestand aus fünf Phasen der Entwicklung und ist in Form eines Meilensteinplans niedergeschrieben. Er befindet sich im Anhang unter Abschnitt \ref{Meilensteinplan}.

Die erste Phase galt der Anforderungsanalyse. 
Wie bereits erwähnt hatten wir zu Beginn eine genaue Vorstellung der Anwendung, welche nun in eine anschauliche Form gebracht werden musste. 
Anschließend folgte die Evaluation der Technologien und Methoden, wie beispielsweise die Datenhaltung innerhalb der App oder die Wahl der AR-Schnittstelle.
Der dritte Meilenstein umfasste das Aufsetzen der Entwicklungsumgebung und Versionsverwaltung, sowie die Aufteilung der Zuständigkeiten.
Im nächsten Schritt wurde das Frontend der Anwendung in Form von GUI-Mockups entworfen und das Backend mithilfe von Klassendiagrammen strukturiert.
In der letzten Phase widmeten wir uns schließlich der Implementierung.
