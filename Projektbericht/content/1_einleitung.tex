\chapter{Einleitung}\label{ch:einleitung}
Der vorliegende Projektbericht dokumentiert die Konzeption und Entwicklung der Lernapplikation \glqq TraWo\grqq, die im Rahmen eines IT-Projektes an der TH-Nürnberg im Bachelorstudiengang Medieninformatik über den Zeitraum Sommersemester 2019 und Wintersemester 2019/2020 durchgeführt wurden.

Der Bericht fasst die erbrachte Teamleistung der Studenten Selina Feitl, Philipp Jäger, Eugen Antonenko und Selin Öztürk zusammen. Zu Beginn wird die eigenständig erarbeitete Projektidee vorgestellt. Dabei wird der Gedankengang hinter der Idee der Applikation sowie die Arbeitsweise des Teams erläutert. 
Da den Großteil der Arbeit die Konzeptionsphase einnahm, sind sämtliche inhaltliche und grafische Anforderungen an die Anwendung zusammengefasst worden. Gemeinsam mit den genutzten Technologien, werden sie als Entwurf des Projektes präsentiert.
Nach einer Detailerklärung der, für den Arbeitsprozess relevanten, Funktionen der  verwendeten Frameworks und Engines, folgt eine ausführliche Beschreibung der Realisierung der Software. Der Fokus liegt dabei auf dem Zusammenspiel zwischen den einzelnen Komponenten, und wie sie eine vollständige und funktionierende App schaffen.
Abschließend folgt ein Fazit des Teams über den Verlauf und das Ergebnis des Projektes. Im Anhang des Berichts lassen sich zudem relevante Artefakte, die im Laufe der Arbeit entstanden sind, finden.

